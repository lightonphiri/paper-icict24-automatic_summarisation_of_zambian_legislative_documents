\documentclass[conference]{IEEEtran}
\IEEEoverridecommandlockouts
% The preceding line is only needed to identify funding in the first footnote. If that is unneeded, please comment it out.
%Template version as of 6/27/2024

\usepackage{cite}
\usepackage{amsmath,amssymb,amsfonts}
\usepackage{algorithmic}
\usepackage{graphicx}
\usepackage{textcomp}
\usepackage{xcolor}
\def\BibTeX{{\rm B\kern-.05em{\sc i\kern-.025em b}\kern-.08em
    T\kern-.1667em\lower.7ex\hbox{E}\kern-.125emX}}
%
%
% Sepetmber 23 2024
\usepackage{cleveref}


\begin{document}

\title{Accessible Legislative Information: Automatic Summarisation of Zambian Legislative Documents}

\author{\IEEEauthorblockN{1\textsuperscript{st} FName1 LName1}
\IEEEauthorblockA{\textit{dept. name of organization (of Aff.)} \\
\textit{name of organization (of Aff.)}\\
City, Country \\
fname1.lname1@institution.xx}
\and
\IEEEauthorblockN{2\textsuperscript{nd} FName2 LName2}
\IEEEauthorblockA{\textit{dept. name of organization (of Aff.)} \\
\textit{name of organization (of Aff.)}\\
City, Country \\
fname2.lname2@institution.xx}
\and
\IEEEauthorblockN{3\textsuperscript{rd} FName3 LName3}
\IEEEauthorblockA{\textit{dept. name of organization (of Aff.)} \\
\textit{name of organization (of Aff.)}\\
City, Country \\
fname3.lname3@institution.xx}
\and
\IEEEauthorblockN{4\textsuperscript{th} FName4 LName4}
\IEEEauthorblockA{\textit{dept. name of organization (of Aff.)} \\
\textit{name of organization (of Aff.)}\\
City, Country \\
fname4.lname4@institution.xx}
\and
\IEEEauthorblockN{5\textsuperscript{th} FName5 LName5}
\IEEEauthorblockA{\textit{dept. name of organization (of Aff.)} \\
\textit{name of organization (of Aff.)}\\
City, Country \\
fname5.lname5@institution.xx}
}

\maketitle

\begin{abstract}
The National Assembly of Zambia produces a number of important legislative documents which are publicly accessible via its Website. One of the documents published by the National Assembly of Zambia are Bills and Acts, which all form the Laws of Zambia. The Acts and Bills are ideally meants to be open and accessible to the general citizenery, however, prior studies conducted have highlighted the lack of ease of access and difficulties with interpretation of such documents and the two main barriers to enabling open and accessible legal information. This paper presents a potentially viable solution to addressing the chellenges with facilitating open and accessible legislative documents by leveraging the use of Natural Language Processing techniques for automatically summarising legislative documents. Specifically, the study was aimed at examinting barriers faced by individuals when comprehending legislative documents and, additionally, determining the feasibility of implementing NLP modules capable of generating concise summaries of legislative documents. In order to understand challenges faced when comprehending legal documents, 150 undergraduate students were sampled from the University of Zambia using random sampling. To determine the feasibility of using NLP techniques to provide concise summaries of legislative documents, two (2) NLP models—an abstractive summarisation model and extractive summarisation model. A human evaluation strategy was used to perform a comparative evaluation of the two (2) NLP models, in order to determine the more effective approach. A significant portion—approximately 74.29\%—of participants reported 'Never' (33.99\%) or 'Rarely' (40.3\%) engaging with legislative documents. In contrast, (25.71\%) indicated frequent or very frequent interaction. This distribution underscores a significant gap in familiarity and engagement with Zambian legislative materials among the study participants. In assessing participants' overall understanding of legal documents, the majority (43.5\%) expressed a neutral perception, suggesting that they found these documents neither easy nor hard to understand. The majority of the human evaluators had a preference for the abstractive summarisation model, indicating that its brevity, simplicity, and directness as reasons for their choice. In addition, the results of the abstractive summarisation model were stated as being easier to understand..
\end{abstract}

\begin{IEEEkeywords}
component, formatting, style, styling, insert.
\end{IEEEkeywords}

\section{Introduction}
\label{sec:introduction}
The National Assembly of Zambia is mandated by law to “To execute the legislative, oversight, representative and budgetary functions for enhanced democratic governance” \cite{NationalAssembly2023Objectives}. In the 2022-2026 strategic plan \cite{NationalAssembly2021Strategic}, the “Strategic Objective 2.2” aims to “Enhance Public Perceptions of the National Assembly” by making parliament open and accessible to the public and, additionally strengthening ICT platforms for public engagement.

While parliament, and entities such as the Zambia Legal Information Institute (ZambiaLII) \cite{ZambiaLII2024Website}, publicly makes available important legislation, interpretation of the documents is problematic due to the size of the documents and the vocabulary used. Masson and Tahir report that the barriers associated with providing open and accessible legal information relies on two factors: ease of access and the capacity to interpret the documents \cite{Masson2016Legal}.

This paper is organised as follows: \Cref{sec:introduction} provides context and background information associated with the studies conducted; \Cref{sec:related_work} comprehensively discusses existing work; \Cref{sec:methodology} outlines the methodological approaches employed when conducting the studies; \Cref{sec:results_and_discussion} discusses the findings and, finally, \Cref{sec:conclusion} outlines concluding remarks and potential future work.

\section{Related Work}
\label{sec:related_work}
Xxxxx xxxxx xxxxx  xxxxx  xxxxx  xxxxx  xxxxx  xxxxx  xxxxx  xxxxx  xxxxx  xxxxx  xxxxx  xxxxx  xxxxx  xxxxx  xxxxx  xxxxx  xxxxx  xxxxx  xxxxx  xxxxx  xxxxx  xxxxx  xxxxx  xxxxx  xxxxx  xxxxx  xxxxx  xxxxx  xxxxx  xxxxx  xxxxx  xxxxx  xxxxx  xxxxx  xxxxx  xxxxx  xxxxx  xxxxx  xxxxx  xxxxx  xxxxx  xxxxx  xxxxx  xxxxx  xxxxx  xxxxx  xxxxx  xxxxx  xxxxx  xxxxx  xxxxx  xxxxx  xxxxx  xxxxx  xxxxx  xxxxx  xxxxx  xxxxx  xxxxx  xxxxx  xxxxx  xxxxx  xxxxx  xxxxx  xxxxx  xxxxx  xxxxx  xxxxx  xxxxx  xxxxx  xxxxx  xxxxx  xxxxx.

\section{Methodology}
\label{sec:methodology}
Xxxxx xxxxx xxxxx  xxxxx  xxxxx  xxxxx  xxxxx  xxxxx  xxxxx  xxxxx  xxxxx  xxxxx  xxxxx  xxxxx  xxxxx  xxxxx  xxxxx  xxxxx  xxxxx  xxxxx  xxxxx  xxxxx  xxxxx  xxxxx  xxxxx  xxxxx  xxxxx  xxxxx  xxxxx  xxxxx  xxxxx  xxxxx  xxxxx  xxxxx  xxxxx  xxxxx  xxxxx  xxxxx  xxxxx  xxxxx  xxxxx  xxxxx  xxxxx  xxxxx  xxxxx  xxxxx  xxxxx  xxxxx  xxxxx  xxxxx  xxxxx  xxxxx  xxxxx  xxxxx  xxxxx  xxxxx  xxxxx  xxxxx  xxxxx  xxxxx  xxxxx  xxxxx  xxxxx  xxxxx  xxxxx  xxxxx  xxxxx  xxxxx  xxxxx  xxxxx  xxxxx  xxxxx  xxxxx  xxxxx  xxxxx.

\section{Results and Discussion}
\label{sec:results_and_discussion}
Xxxxx xxxxx xxxxx  xxxxx  xxxxx  xxxxx  xxxxx  xxxxx  xxxxx  xxxxx  xxxxx  xxxxx  xxxxx  xxxxx  xxxxx  xxxxx  xxxxx  xxxxx  xxxxx  xxxxx  xxxxx  xxxxx  xxxxx  xxxxx  xxxxx  xxxxx  xxxxx  xxxxx  xxxxx  xxxxx  xxxxx  xxxxx  xxxxx  xxxxx  xxxxx  xxxxx  xxxxx  xxxxx  xxxxx  xxxxx  xxxxx  xxxxx  xxxxx  xxxxx  xxxxx  xxxxx  xxxxx  xxxxx  xxxxx  xxxxx  xxxxx  xxxxx  xxxxx  xxxxx  xxxxx  xxxxx  xxxxx  xxxxx  xxxxx  xxxxx  xxxxx  xxxxx  xxxxx  xxxxx  xxxxx  xxxxx  xxxxx  xxxxx  xxxxx  xxxxx  xxxxx  xxxxx  xxxxx  xxxxx  xxxxx.

\section{Conclusions and Future Work}
\label{sec:conclusion}
Xxxxx xxxxx xxxxx  xxxxx  xxxxx  xxxxx  xxxxx  xxxxx  xxxxx  xxxxx  xxxxx  xxxxx  xxxxx  xxxxx  xxxxx  xxxxx  xxxxx  xxxxx  xxxxx  xxxxx  xxxxx  xxxxx  xxxxx  xxxxx  xxxxx  xxxxx  xxxxx  xxxxx  xxxxx  xxxxx  xxxxx  xxxxx  xxxxx  xxxxx  xxxxx  xxxxx  xxxxx  xxxxx  xxxxx  xxxxx  xxxxx  xxxxx  xxxxx  xxxxx  xxxxx  xxxxx  xxxxx  xxxxx  xxxxx  xxxxx  xxxxx  xxxxx  xxxxx  xxxxx  xxxxx  xxxxx  xxxxx  xxxxx  xxxxx  xxxxx  xxxxx  xxxxx  xxxxx  xxxxx  xxxxx  xxxxx  xxxxx  xxxxx  xxxxx  xxxxx  xxxxx  xxxxx  xxxxx  xxxxx  xxxxx.


% % % % % \section*{References}

\bibliographystyle{IEEEtran}
\bibliography{paper-icict24-automatic_summarisation_of_zambian_legislative_documents}

\end{document}
